\documentclass[11pt]{article}
\usepackage{amsmath,textcomp,amssymb,geometry,graphicx,enumerate}
\usepackage{algorithm} % Boxes/formatting around algorithms
\usepackage[noend]{algpseudocode} % Algorithms
\usepackage{hyperref}
\hypersetup{
    colorlinks=true,
    linkcolor=blue,
    filecolor=magenta,
    urlcolor=blue,
}

\def\Name{Samuel Cuthbertson}  % Your name
\def\Homework{2} % Number of Homework
\def\Session{Spring 2015}


\title{Discrete Structures --- Problem Set \Homework}
\author{\Name}
\markboth{\Homework\ \Name}{\Homework\ \Name}
\pagestyle{myheadings}
\date{}

\newenvironment{qparts}{\begin{enumerate}[{(}a{)}]}{\end{enumerate}}
\def\endproofmark{$\Box$}
\newenvironment{proof}{\par{\bf Proof}:}{\endproofmark\smallskip}

\textheight=9in
\textwidth=6.5in
\topmargin=-.75in
\oddsidemargin=0.25in
\evensidemargin=0.25in


\begin{document}
\maketitle

\section*{Problem 1}
The ``pile method'' can be rewritten as $pile(n) = \sum\limits_{i=1}^{n-1}(i)$ The proof of this is as follows:
\[ Given \ pile(1) = 1, \ and \ pile(n) = \sum\limits_{i=1}^{n-1}(i) \ for \ all \ n <= 10 \]
\[ Then \ pile(k+1) =  \sum\limits_{i=1}^{k}(i) = pile(k) + (k) = \sum\limits_{i=1}^{k-1}(i) + k \ for \ all \ n \in \\R \]
Which is a valid strong induction proof, since $ \sum\limits_{i=1}^{k}(i) = \sum\limits_{i=1}^{k-1}(i) + k$.

\section{Problem 2}
\begin{qparts}
\item
Using the same method as described in the phrasing of question 2, we can obtain
\[ J\ mod\ 3 = 2\qquad J\ mod\ 5=0\qquad J\ mod\ 7=3\]
\[ 3*(-3) + 5*(2) = 1 \implies 3*(-3)*(0) + 5 * (2) * (2) = 20 \]
And now solving for the final answer given that product and J mod 7 = 3
\[J\ mod\ 15=20\qquad J\ mod\ 7=3\]
\[ 15*(-\frac{4}{17}) + 5*(\frac{11}{17}) = 1 \implies 15*(-\frac{4}{17})*(3) + 7 * (\frac{11}{17}) * (20) = 80 \]

\item
And for the general, we can use the exact same method:
\[ J\ mod\ 11 = 1\qquad J\ mod\ 13=8\qquad J\ mod\ 17=2 \]
\[ 11*(1) + 13*(-\frac{10}{13}) = 1 \implies 11*(1)*(8) + 13*(\frac{-10}{13})*(1) = 78 \]
\[J\ mod\ 143=78\qquad J\ mod\ 17=2\]
\[ 143*(-\frac{347}{988}) + 17*(\frac{229}{76}) = 1 \implies 143*(-\frac{347}{988})*2 + 17*(\frac{229}{76})*(78) = 3895 \]

\end{qparts}

\section{Problem 3}
\begin{qparts}
\item
We can say that all the exponents must either be 3 or be 0, since any $n^3$ can be factored out into some number of primes $p_1^3 * p_2^3 * ... * p_i^3$.

\item
From staring at the list here (https://www.math.upenn.edu/~deturck/m170/wk2/divisors.html), the pattern that pops out is $\sum\limits_{i=1}^{n}(2*e_n)$

\end{qparts}

\section{Problem 4}
\begin{qparts}
\item
This is the definition of Euler's Totient. In the example, 5 and 7 are distinct primes, so the totient is equal to (5-1)(7-1).

\item
If $m=p^k$, the numbers that have a common factor with m are all the multiples of p. There are $p^{k-1}$ multiples, so the totient of $p^k$ is $p^k - p^{k-1}$

\end{qparts}

\section{Problem 5}
	Note that: \\
Regular 3-gons: 1 \\
Regular 4-gons: 1 \\
Regular 5-gons: 2 \\
Regular 6-gons: 1 \\
Regular 7-gons: 3 \\
Regular 8-gons: 2 \\
Regular 9-gons: 4 \\
Regular 10-gons: 2 \\
Regular 11-gons: 5 \\

Thus, the number of distinct n-gons is \textbf{$\frac{n-1}{2}$} if n is odd and \textbf{$\frac{n-2}{2}$} if n is even, and n-gon(47) = $\frac{47-1}{2} = \frac{46}{2} = 23$.
\end{document}
